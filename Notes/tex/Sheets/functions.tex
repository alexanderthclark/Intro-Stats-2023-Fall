Google Sheets offers hundreds of functions across several categories. We are especially interested in the statistical, math, and logical categories.\footnote{Find the full function list at \link{https://support.google.com/docs/table/25273?hl=en}{https://support.google.com/docs/table/25273?hl=en}.}


\emph{in progress}

\subsection{Mathematical and Statistical}

Below is a selection of commonly used function. Their purpose should be apparent from the name. To me, the trickiest thing is remembering to use \code{AVERAGE} instead of trying \code{MEAN}, which doesn't exist. 

\begin{center}

Descriptive Statistics\\
\begin{tabular}{rrr}
\toprule
Function & Sample Usage & Notes \\
\midrule
\link{https://support.google.com/docs/answer/3093615?sjid=720707396607486715-NA}{\code{AVERAGE}} & \code{AVERAGE(A1:A10)} & \\ % AVERAGE
\link{https://support.google.com/docs/answer/3093990?sjid=18107305809463122801-NA}{\code{CORREL}} & \code{CORREL(A1:A10, B1:B10)} & Correlation, same as \code{PEARSON}. \\ % CORREL
\link{https://support.google.com/docs/answer/3093993?hl=en&sjid=18107305809463122801-NA}{\code{COVAR}} & \code{COVAR(A1:A10, B1:B10)} & \\ % COVAR
\link{https://support.google.com/docs/answer/3094025}{\code{MEDIAN}} & \code{MEDIAN(A1:A10)} & \\ % MEDIAN
\link{https://support.google.com/docs/answer/3267350}{\code{PERCENTILE}} & \code{PERCENTILE(A1:A10, 0.5)} & \\ % PERCENTILE
\link{https://support.google.com/docs/answer/3094063}{\code{VAR}} & \code{VAR(A1:A10)} & This is SD${^{+}}^2$.\\ % VAR
\link{https://support.google.com/docs/answer/3094113?sjid=18107305809463122801-NA}{\code{VARP}} & \code{VARP(A1:A10)} & \\ % VARP
\link{https://support.google.com/docs/answer/3093669}{\code{SUM}} & \code{SUM(A1:A10)} & \\ % SUM
\link{https://support.google.com/docs/answer/3094054?sjid=18107305809463122801-NA}{\code{STDEV}} & \code{STDEV(A1:A10)} & This is SD$^{+}$, not SD, per \cite{freedman2007statistics}. \\ % STDEV
\link{https://support.google.com/docs/answer/3094105}{\code{STDEVP}} & \code{STDEVP(A1:A10)} & This is SD, per \cite{freedman2007statistics}. \\ % STDEVP
\bottomrule
\end{tabular}
\end{center}


Next, we have some common probability distributions and the accompanying functions. You might also see the functions named differently, \code{NORM.DIST} instead of \code{NORMDIST} for example. Sometimes they are the same and sometimes they are slightly different. For example, \code{CHIDIST} calculates the right-tail probability and \code{CHISQ.DIST} calculates the left tail probability. The \code{cumulative} parameter should be set to \code{True} or \code{False}.

\begin{center}

Probability Distributions \\ 
\begin{tabular}{rrr}
\toprule
Function & Sample Usage & Notes \\
\midrule
\link{https://support.google.com/docs/answer/3094021}{\code{NORMDIST}} & \code{NORMDIST(x, mean, std dev, cumulative)} & \\ % NORM.DIST
\link{https://support.google.com/docs/answer/3093987}{\code{BINOMDIST}} & \code{BINOMDIST(num successes, trials, probability success, cumulative)} & \\ % BINOM.DIST
\link{https://support.google.com/docs/answer/7003346}{\code{CHIDIST}} & \code{CHIDIST(x, degrees of freedom)} & right-tailed \\ % CHIDIST
\link{https://support.google.com/docs/answer/7003347}{\code{CHISQ.DIST}} & \code{CHISQ.DIST(x, degrees of freedom, cumulative)} & left-tailed \\ % CHISQ.DIST
\link{https://support.google.com/docs/answer/3295914}{\code{TDIST}} & \code{TDIST(x, degrees of freedom, tails)} & \\ % T.DIST
\link{https://support.google.com/docs/answer/3094022}{\code{NORMINV}} & \code{NORMINV(probability, mean, std dev)} & \\ % NORMINV
%\code{BINOM.INV} & \code{BINOM.INV(trials, probability success, alpha)} & \\ % BINOM.INV
%\link{https://support.google.com/docs/answer/3094106}{\code{CHISQ.INV}} & \code{CHISQ.INV(probability, degrees of freedom)} & \\ % CHISQ.INV
\link{https://support.google.com/docs/answer/6055811}{\code{TINV}} & \code{TINV(probability, degrees of freedom)} & Two-tailed inverse \\ % T.INV
\bottomrule
\end{tabular}


\end{center}

\subsection{Logical}


\begin{center}

Logical Functions \\ 
\begin{tabular}{rrr}
\toprule
Function & Syntax & Example \\
\midrule
\link{https://support.google.com/docs/answer/3093364}{\code{IF}} & \code{IF(logical_expression, value_if_true, value_if_false)} & \code{IF(A1 > 0, A1, 0)} \\ % IF
\link{https://support.google.com/docs/answer/3093301}{\code{AND}} & \code{AND(logical_expression, [logical_expressions, ...])} & \code{AND(A1 > 0, A1 < 2)}  \\ % AND
\link{https://support.google.com/docs/answer/3093306}{\code{OR}} & \code{OR(logical_expression, [logical_expressions, ...])} & \code{OR(A1 > 2, A1 < -2)} \\ % OR
\link{https://support.google.com/docs/answer/3256529}{\code{AVERAGEIF}} & \code{AVERAGEIF(criteria_range, criterion, [average_range])} & \code{AVERAGEIF(A:A, ">0")}  \\ % AVERAGEIF
\link{https://support.google.com/docs/answer/3093583}{\code{SUMIF}} & \code{SUMIF(criteria_range, criterion, [sum_range])} & \code{SUMIF(A:A, "taxi", B:B)} \\ % SUMIF

\bottomrule
\end{tabular}

\end{center}



\subsection{Lookup Functions}

\begin{tabular}{rrr}
\toprule
Function & Syntax & Example \\
\midrule
\link{https://support.google.com/docs/answer/12405947}{\code{XLOOKUP}} & \code{XLOOKUP(search_key, lookup_range, range_result} \scalebox{0.5}{\code{, missing_value, match_mode, search_mode)}} & \code{XLOOKUP("Shaq", A:A, B:B)} \\ % IF

\bottomrule
\end{tabular}
